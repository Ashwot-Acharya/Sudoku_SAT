\documentclass{beamer}

\usetheme{CambridgeUS}
\usecolortheme{beaver}
\usefonttheme{professionalfonts}
\setbeamertemplate{navigation symbols}{}
\setbeamertemplate{caption}[numbered]

\usepackage{graphicx}
\usepackage{amsmath}
\usepackage{amssymb}

\title[SAT Encoding of Sudoku]{Optimized CNF Encoding for Sudoku}
\author{Ashwot Acharya , Lakki Thapa , Bishesh Bohora , Supreme Chaudhary}
\institute{Your University}
\date{\today}

\begin{document}

%------------------------------------------------
\begin{frame}
\titlepage
\end{frame}
%------------------------------------------------

\begin{frame}{Motivation}
Sudoku can be encoded as a SAT problem.

\begin{itemize}
    \item Each cell $(r,c)$ contains a value $v \in \{1,\dots,n\}$
    \item Represent using Boolean variables:
    \[
        x_{r,c,v}
    \]
    \item True iff cell $(r,c)$ has value $v$
\end{itemize}

Goal:
\begin{itemize}
    \item Convert Sudoku constraints into CNF
    \item Solve using a SAT solver
\end{itemize}

\end{frame}

%------------------------------------------------

\begin{frame}{Variable Encoding}

For an $n \times n$ Sudoku:

\[
V = \{(r,c,v) \mid 1 \le r,c,v \le n \}
\]

Total variables:
\[
n^3
\]

Example (9×9):
\[
9^3 = 729 \text{ variables}
\]

\end{frame}

%------------------------------------------------

\begin{frame}{Core Constraints}

\textbf{Cell constraints}
\begin{itemize}
    \item Each cell has at least one value
    \item Each cell has at most one value
\end{itemize}

\textbf{Row constraints}
\begin{itemize}
    \item Each value appears once per row
\end{itemize}

\textbf{Column constraints}
\begin{itemize}
    \item Each value appears once per column
\end{itemize}

\textbf{Block constraints}
\begin{itemize}
    \item Each value appears once per block
\end{itemize}

\end{frame}

%------------------------------------------------

\begin{frame}{Previous Encodings}

From:
\textit{Optimized CNF Encoding for Sudoku Puzzles} \\
Kwon \& Jain :contentReference[oaicite:1]{index=1}

Three encodings:

\begin{itemize}
    \item Minimal
    \item Efficient
    \item Extended
\end{itemize}

All produce:
\[
O(n^4) \text{ clauses}
\]

Problem:
\begin{itemize}
    \item Very large CNF for large grids (e.g., 81×81)
\end{itemize}

\end{frame}

%------------------------------------------------

\begin{frame}{Extended Encoding}

Extended encoding includes:

\begin{itemize}
    \item Cell definedness
    \item Cell uniqueness
    \item Row definedness
    \item Row uniqueness
    \item Column definedness
    \item Column uniqueness
    \item Block definedness
    \item Block uniqueness
\end{itemize}

Produces many redundant clauses.

\end{frame}

%------------------------------------------------

\begin{frame}{Optimized Encoding Idea}

Key observation:

Fixed cells imply:
\begin{itemize}
    \item Some variables are already TRUE
    \item Many others are forced FALSE
\end{itemize}

Therefore:
\begin{itemize}
    \item Remove redundant literals
    \item Remove satisfied clauses
\end{itemize}

This reduces:
\begin{itemize}
    \item Number of variables
    \item Number of clauses
\end{itemize}

\end{frame}

%------------------------------------------------

\begin{frame}{Variable Partition}

Variables are partitioned:

\[
V = V^+ \cup V^- \cup V'
\]

\begin{itemize}
    \item $V^+$ : true variables (fixed cells)
    \item $V^-$ : forced false variables
    \item $V'$ : remaining unknown variables
\end{itemize}

Reduction operators eliminate:
\begin{itemize}
    \item Clauses already satisfied
    \item Literals known to be false
\end{itemize}

\end{frame}

%------------------------------------------------

\begin{frame}{Effect of Optimization}

From experimental results :contentReference[oaicite:2]{index=2}:

\begin{itemize}
    \item Variables reduced up to 12×
    \item Clauses reduced up to 79×
    \item Significant SAT solving speedup
\end{itemize}

Example (81×81):

\begin{itemize}
    \item Extended: 85M clauses
    \item Optimized: 266K clauses
\end{itemize}

\end{frame}

%------------------------------------------------

\begin{frame}{Our Implementation}

We implemented:

\begin{itemize}
    \item CNF encoding in DIMACS format
    \item Optimized preprocessing for fixed cells
    \item Benchmarks on:
    \begin{itemize}
        \item 9×9 (17 clues)
        \item 9×9 (20+ clues)
        \item Larger grids
    \end{itemize}
\end{itemize}

\end{frame}

%------------------------------------------------

\begin{frame}{Benchmark Results – Runtime}

\begin{center}
\includegraphics[width=0.85\linewidth]{../runtime_plot.pdf}
\end{center}

\end{frame}

%------------------------------------------------

\begin{frame}{Benchmark Results – Clause Count}

\begin{center}
\includegraphics[width=0.85\linewidth]{../clause_plot.pdf}
\end{center}

\end{frame}

%------------------------------------------------

\begin{frame}{Benchmark Results – Variable Count}

\begin{center}
\includegraphics[width=0.85\linewidth]{../variable_plot.pdf}
\end{center}

\end{frame}

%------------------------------------------------

\begin{frame}{Observations}

\begin{itemize}
    \item More clues $\Rightarrow$ fewer variables
    \item More clues $\Rightarrow$ fewer clauses
    \item Optimized encoding significantly reduces CNF size
    \item SAT solver runtime strongly correlates with clause count
\end{itemize}

\end{frame}

%------------------------------------------------

\begin{frame}{Conclusion}

\begin{itemize}
    \item Sudoku naturally maps to SAT
    \item Naive encodings produce large CNF formulas
    \item Using fixed-cell reduction:
    \begin{itemize}
        \item Shrinks formula
        \item Improves solver performance
    \end{itemize}
    \item Effective for larger Sudoku sizes
\end{itemize}

\end{frame}

%------------------------------------------------

\begin{frame}{References}

\small

Kwon, G., Jain, H. \\
\textit{Optimized CNF Encoding for Sudoku Puzzles} :contentReference[oaicite:3]{index=3}

Lynce, I., Ouaknine, J. \\
\textit{Sudoku as a SAT Problem}

Moskewicz et al. \\
\textit{Chaff: Engineering an Efficient SAT Solver}

\end{frame}

%------------------------------------------------

\end{document}
