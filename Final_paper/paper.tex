\documentclass{article}
\usepackage{amsmath, amssymb}
\usepackage{graphicx}
\usepackage{geometry}
\usepackage{braket}
\usepackage[backend=biber,style=authoryear]{biblatex}
\addbibresource{references.bib}
\geometry{a4paper, margin=1in}
\usepackage{titlesec}
\usepackage{tikz} 
\geometry{
  left=1.2in,
  right=1.2in,
  top=1.1in,
  bottom=1.1in
}
\usetikzlibrary{arrows.meta, positioning}
\titleformat{\section}{\large\bfseries}{\thesection}{1em}{}
\titleformat{\subsection}{\normalsize\bfseries}{\thesubsection}{1em}{}

\date{\today}
\begin{document}
\hbadness=10000
\begin{titlepage}
    \begin{center}
      
        \Large\bfseries
        SAT Based approach to solving sudoku\\[2cm]
        
        \normalsize
        THIRD YEAR PROJECT REPORT\\[2ex]
        
        \small
        SUBMITTED IN PARTIAL FULFILLMENT OF THE REQUIREMENTS FOR\\
        THE DEGREE OF BSC. IN COMPUTATIONAL MATHEMATICS\\[2cm]
        
        \normalsize
        BY\\[2ex]
        \textbf{Ashwot Acharya, Bishesh Bohora, Supreme Chaudhary, Lakki Thapa}\\[2cm]
        
        
        DEPARTMENT OF MATHEMATICS\\
        KATHMANDU UNIVERSITY\\
        DHULIKHEL, NEPAL\\[2ex]
                
        \vfill
    \end{center}
\end{titlepage}
\section{CNF Conversion for  $  4 \times 4 $  sudoku } 

\subsection{What is CNF?}
A boolean formula is said to be in a conjective normal form if it is a conjuction of clauses , where each clause is a disjunction of literals.\cite{HuthRyan2004}

To show how a sudoku is converted to a Conjective Normal Form (CNF) let us consider a basic $4 \times 4 $ sudoku like the following: 

\[
\renewcommand{\arraystretch}{2.5}
\begin{array}{|p{2cm}|p{2cm}||p{2cm}|p{2cm}|}
\hline
 & 2 &  &  \\
\hline
 &  & 3 &  \\
\hline\hline
 & 3  &  &  \\
\hline
 &  & 1 &  \\
\hline
\end{array}
\]
\subsection{Basic representation} 
Each cell in a $4 \times 4$ sudoku has a total of 4 different possible values 
for the ease of representation let us consider the following : \\
$i$:  row number  \\
$j$: column number \\ 
$k$: possible input value  \\  
$ i, j , k \in \{ 1 , 2 , 3, 4 \} $ \\ 
And hence any cell  $x $ can be represented as : 
$ x_{ijk} $

\subsection{Clause for each cells} 
Since there only exist 1 input in each cell it can be written in terms of logic as : \\ 
\[ ( x_{ij1} \vee x_{ij2} \vee x_{ij3} \vee x_{ij4}) \]
so for first cell we can represent the logic as : 
   $ (x_{111} \vee x_{112} \vee x_{113} \vee x_{114} ) $ \\ 
rewriting for each cell we get the following: 
\[
\[
\renewcommand{\arraystretch}{2.5}
\begin{array}{|p{2cm}|p{2cm}||p{2cm}|p{2cm}|}
\hline
$\bigvee_{k=1}^{4} x_{11k}$ & $x_{122}$ & $\bigvee_{k=1}^{4} x_{13k}$ & $\bigvee_{k=1}^{4} x_{14k}$ \\
\hline
$\bigvee_{k=1}^{4} x_{21k}$ & $\bigvee_{k=1}^{4} x_{22k}$ & $x_{233}$ & $\bigvee_{k=1}^{4} x_{11k}$ \\
\hline\hline
$\bigvee_{k=1}^{4} x_{32k}$ & $ x_{323} $ & $\bigvee_{k=1}^{4} x_{33k}$ & $\bigvee_{k=1}^{4} x_{34k}$ \\
\hline
$\bigvee_{k=1}^{4} x_{41k}$ & $\bigvee_{k=1}^{4} x_{42k}$ & $x_{431}$ & $\bigvee_{k=1}^{4} x_{44k}$ \\
\hline
\end{array}
\]

Now Since each cell has at most 1 digit  we can represent in CNF form in the following way: \\ 

\[
\begin{aligned}
\neg (x_{111} \land x_{112}) 
&\equiv (\neg x_{111} \vee \neg x_{112}) 
&\hfill \text{(by De Morgan’s law)}
\end{aligned}
\] 
\clearpage
So Applying For each possible values in first row first column we get the following 6 clauses: 
$$ (\neg x_{111} \vee\neg x_{112}  ) $$ 
$$ (\neg x_{111} \vee\neg x_{113}  ) $$  
$$ (\neg x_{111} \vee\neg x_{114}  ) $$  
$$ (\neg x_{112} \vee\neg x_{113}  ) $$  
$$ (\neg x_{112} \vee\neg x_{114}  ) $$  
$$ (\neg x_{113} \vee\neg x_{114}  ) $$  

The above is repeatitive pattern for each cell of the sudoku for example in  cell 2 we will get the following 
$$ (\neg x_{121} \vee \neg x_{122}  ) $$ 
$$ (\neg x_{121} \vee \neg x_{123}  ) $$  
$$ (\neg x_{121} \vee \neg x_{124}  ) $$  
$$ (\neg x_{122} \vee \neg x_{123}  ) $$  
$$ (\neg x_{122} \vee \neg x_{124}  ) $$  
$$ (\neg x_{123} \vee \neg x_{124}  ) $$  and so forth for the remaining cells with total of $ 16 \times 6  = 96$ clauses

The values in row and column of sudoku are unique i.e non repeatitive  so row and column can have 1 to 4 in a non repetitive pattern the same follows for the sub grid of the sudoku .
\subsection{uniqueness of row and column} 
\subsubsection{For row} 
One digit on row must always appear only one in that row , to represent that 
for uniqueness of digit "1" : 
$$ \neg x_{111} \vee \neg x_{121}  $$
$$ \neg x_{111} \vee  \neg x_{131}  $$
$$ \neg x_{111} \vee \neg x_{141}  $$
$$ \neg x_{121} \vee \neg x_{131} $$
$$ \neg x_{121} \vee \neg x_{141}  $$
$$ \neg x_{131} \vee \neg x_{141}  $$

uniqueness of digit "2" :
$$  \neg x_{112} \vee \neg x_{122}  $$
$$ \neg x_{112} \vee  \neg x_{132}  $$
$$ \neg x_{112} \vee \neg x_{142}  $$
$$ \neg x_{122} \vee \neg x_{132} $$
$$ \neg x_{122} \vee \neg x_{142}  $$
$$ \neg x_{132} \vee \neg x_{142}  $$


Similarly follows for 3 and 4 and for all four rows
\subsubsection{For Column} 

for uniqueness of digit "1"  in the first column , 
for uniqueness of digit "1" : 
$$ \neg x_{111} \vee \neg x_{211}  $$
$$ \neg x_{111} \vee  \neg x_{311}  $$
$$ \neg x_{111} \vee \neg x_{411}  $$
$$ \neg x_{211} \vee \neg x_{311} $$
$$ \neg x_{211} \vee \neg x_{411}  $$
$$ \neg x_{311} \vee \neg x_{411}  $$

uniqueness of digit "2" in the first column :
$$  \neg x_{112} \vee \neg x_{212}  $$
$$ \neg x_{112} \vee  \neg x_{312}  $$
$$ \neg x_{112} \vee \neg x_{412}  $$
$$ \neg x_{212} \vee \neg x_{312} $$
$$ \neg x_{212} \vee \neg x_{412}  $$
$$ \neg x_{312} \vee \neg x_{412}  $$

and the same goes for digits '3' and digits  '4' and follows for the remaining column 
i.e for each column there is $6 \times 4 = 24 $ statements, and for column as a whole there are 96 statements the same goes for no of clauses for rows.

\subsubsection{for $3 \times 3$ sub grid} 
In each subgrid there can only be one unique digits i.e there is no repetition of the digits in the subgrid. 
So In terms of logic it can be written as : \\
for example no repetition of digit '1' in the first subgrid of $2 \time 2$ 

$$ \neg x_{111} \vee \neg x_{112} $$
$$ \neg x_{111} \vee \neg x_{211} $$ 
$$ \neg x_{111} \vee \neg x_{221} $$ 
$$ \neg x_{121} \vee \neg x_{211} $$ 
$$ \neg x_{121} \vee \neg x_{221} $$ 
$$ \neg x_{211} \vee \neg x_{221} $$

It follows similarly for all digits 2 to 4 in the similar pattern and for each of the subgrids.

\section{Writing of the CNF as a whole} 
The entire length of the CNF of the $ 4 \time 4 $ sudoku will be pretty long but is written by  AND-ing 
of each of the above clauses to create a long boolean Satisfiaility problem . \\ 
A snippet of the CNF form as a whole may be as follows : \\ 
uniqueness of the digit in the first cell 
\[(\neg x_{111} \vee\neg x_{112}  )  \land 
 (\neg x_{111} \vee\neg x_{113}  )  \land
 (\neg x_{111} \vee\neg x_{114}  )   \land
 (\neg x_{112} \vee\neg x_{113}  )  \land
 (\neg x_{112} \vee\neg x_{114}  )  \land 
 (\neg x_{113} \vee\neg x_{114}  ) \land \] \[
(\neg x_{121} \vee \neg x_{122}  )  \land 
 (\neg x_{121} \vee \neg x_{123}  ) \land   
 (\neg x_{121} \vee \neg x_{124}  )  \land 
 (\neg x_{122} \vee \neg x_{123}  )\land 
(\neg x_{122} \vee \neg x_{124}  )  \land 
(\neg x_{123} \vee \neg x_{124}  )  \land 
...
  \]


\printbibliography




\end{document}